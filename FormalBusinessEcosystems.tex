% This is samplepaper.tex, a sample chapter demonstrating the
% LLNCS macro package for Springer Computer Science proceedings;
% Version 2.21 of 2022/01/12
%

\documentclass[runningheads]{llncs}
%
\usepackage[T1]{fontenc}
\usepackage{pdfpages}
\usepackage{tabularx}
\usepackage{booktabs}
\newcolumntype{Y}{>{\raggedright\arraybackslash}X}

% T1 fonts will be used to generate the final print and online PDFs,
% so please use T1 fonts in your manuscript whenever possible.
% Other font encondings may result in incorrect characters.
%
\usepackage{graphicx}
% Used for displaying a sample figure. If possible, figure files should
% be included in EPS format.
%
% If you use the hyperref package, please uncomment the following two lines
% to display URLs in blue roman font according to Springer's eBook style:
%\usepackage{color}
%\renewcommand\UrlFont{\color{blue}\rmfamily}
%\urlstyle{rm}
%
\begin{document}
%
\title{Knowledge Elicitation and Ontology-Based Visualization of Business Ecosystems: A Case Study from the 
Wind Energy Ecosystem}
%
%\titlerunning{Abbreviated paper title}
% If the paper title is too long for the running head, you can set
% an abbreviated paper title here
%
\author{Alican Tüzün\inst{1,2}\orcidID{0009-0009-8017-5487} \and
Georgios Meditskos\inst{1}\orcidID{0000-0003-4242-5245}
}
%
\authorrunning{Tüzün et al.}
% First names are abbreviated in the running head.
% If there are more than two authors, 'et al.' is used.
%
\institute{School of Informatics, Aristotle University of Thessaloniki, Thessaloniki, Greece\and
Josef Ressel Centre for Data-Driven Business Model Innovation, University of Applied Sciences Upper Austria, 
Wehrgrabengasse 1-4, 4400, Steyr, Austria
\email{lncs@springer.com}\\
\url{http://www.springer.com/gp/computer-science/lncs}}
%
\maketitle              % typeset the header of the contribution
%
\begin{abstract}
The abstract should briefly summarize the contents of the paper in
150--250 words.

\keywords{Business Ecosystem  \and Knowledge Representation \and Ontology \and Wind Energy \and Green Energy.}
\end{abstract}
%
%
%
\section{Introduction}

Main idea: Conventional modeling approaches does not leverage semantics, therefore the visualizations gets too complex such that decision makers cannot comprehend.
However with the SPARQL querries, one can easily semantically query the information (e.g traversing the graphs) and visualize whatever needed for the decision maker. With the former 
methodologies it is not possible.

Sub effect: Because of the formality, one can also use the symbolic logic therefore can infer new relationships within the data, which is also not possible with the conventional methods.
Sub effect: Formal and needed representations reduces the ambiguity and the complexity of the data, therefore the decision makers can easily understand the data and make decisions.

Analogy: To go through something with a fine-tooth comb 


\subsection{P1: Challange}
To make critical decisions, organizations need to understand not only their internal operations but also the external environment where 
they operate or may operate. However, visualizing and understanding these complex environment still remains a challenge. And proposed solutions
are nearly to unusable for business developer managers and business strategists, where most lacks semantic interoperability and are not 
based on business understanding but rather on technical understanding.

To cite
Visual decision support for business ecosystem analysis
Visual Analysis of Venture Similarity in Entrepreneurial Ecosystems

\subsection{P2: Business, Wind Energy Ecosystems}
A business ecosystem is a ecosystem that focuses on organicism world view, where Moore directly borrows from biological ecosystems, emphasizing
interdependence, co-evolution and emergent behavior.

not only the businesses but includes other types of organizations
such as non-profits. Wind energy ecosystem on the other hand context specific ecosystem that consists of
organizations that are related to wind energy production, distribution and maintenance. However such ecosystems 
are not easy to understand and analyze due to the complex nature of the interactions between the organizations. Therefore,
to understand this ecosystem, one need to capture the interactions between the organizations and represent them
in a structured way.

\subsection{P3: Knowledge Representation, Ecosystem Knowledge}
One way of structring such structure is to use ontologies. Ontologies are 

\subsection{P4:Ecosystem interactions}

\subsection{P5: }


\begin{itemize}
    \item \textbf{P1: Challenge}
    \item \textbf{P2: Business, Green, Wind Energy Ecosystems}
    \item \textbf{P3: Knowledge Representation, Ecosystem Knowledge}
    \item \textbf{P4: Ecosystem interactions}
    \item \textbf{P5: Related Work and why they are insufficient}
\end{itemize}

\subsection*{\textbf{Research Question}}
\begin{center}
    \textit{How can organizational interactions in the wind energy ecosystem 
systematically captured and explicitized into structured, formal knowledge representations to enable data-driven decision making?}
\end{center}

\section{Methology}

\subsection{Semi-Structured Survey}

\begin{table}[ht]
    \centering
    \caption{Relationships and Theoretical Foundations}
    \label{tab:relationships}
    \begin{tabularx}{\textwidth}{lYY}
    \toprule
    \textbf{Relationship Type} & \textbf{Theoretical Foundation} & \textbf{Logical Charecteristics} \\
    \midrule
    Product \& Service Delivery & 
    Supply Chain Management (Chopra \& Meindl, 2016); 
    Value Chain Analysis (Porter, 1985); 
    Business Ecosystems (Adner, 2017) & 
    Irreflexive, Transitive \\
    \addlinespace
    
    Payment & 
    Business Model Ontology (Osterwalder \& Pigneur, 2005); 
    Value Network Analysis (Allee, 2008); 
    Input-Output Economics (Leontief, 1986) & 
    Irreflexive \\
    \addlinespace
    
    Data & 
    Knowledge-Based View (Grant, 1996); 
    Digital Ecosystem Theory (Tiwana, 2013) &
    Irreflexive \\
    \addlinespace
    
    Information & 
    Knowledge-Based View (Grant, 1996) & 
    Irreflexive \\
    \addlinespace
    
    Collaboration & 
    Resource-Based View (Barney, 1991) & 
    Irreflexive, Symmetric \\
    \addlinespace
    
    Conflict & 
    Stakeholder Theory (Freeman, 1984) & 
    Irreflexive,ASymmetric \\
    \addlinespace
    
    Competition & 
    Porter's Five Forces (Porter, 1979) & 
    Irreflexive, Symmetric \\
    \addlinespace
    
    Coopetition (Implicit) & 
    Coopetition Theory (Brandenburger \& Nalebuff, 1996) & 
    Irreflexive \\
    \bottomrule
    \end{tabularx}
    \smallskip
    \end{table}

\subsection{Term Disambiguation}

\subsection{OWL2 \& Ontological Commitments}

\begin{itemize}
    \item \textbf{ClassAssertion}
    \item \textbf{ClassHierarchyAssertion}
    \item \textbf{ClassDisjointnessAssertion}
    \item \textbf{ObjectPropertyAssertion}
    \item \textbf{PropertyCharacteristicAssertions}
    \item \textbf{Methodological Limitations}
\end{itemize}

\subsection{Query Language}
\begin{itemize}
    \item \textbf{SPARQL}
    \item \textbf{Fuseki Server}
\end{itemize}
\subsection{Relationship Visualization}
\begin{itemize}
    \item \textbf{js and d3.js}
\end{itemize}

\section{Results\&Discussion}

\subsection{Survey Results\&Discussion}

\subsection{Ontology Development}
\subsection{Information Retrieval with Sparql}
\subsection{Visualization Results}

\section{Conclusion}

\section{Appendix}
\appendix
\section{Semi-Structured Survey}
%\includepdf[pages=-]{survey.pdf}
\section{Github Repo}

\bibliographystyle{splncs04}
\bibliography{reference}

\end{document}

